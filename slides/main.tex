\documentclass[12pt]{beamer}
\usepackage[utf8]{inputenc}
\usepackage{hyperref}
\usepackage{graphicx}
\usepackage{comment}

\usetheme{Madrid}
\usecolortheme{beaver}
\setbeamertemplate{enumerate items}[default]

\AtBeginSection[]{
  \begin{frame}
    \vfill\centering
    \usebeamerfont{title}\insertsectionhead\par%
    \vfill
  \end{frame}
}

\title[Git Tutorial]{Git \& GitHub: A Beginner's Guide for Research Collaboration}
\author{Chiatse Wang}
\institute[ISS AS]{Institute of Statistical Science, Academia Sinica}
\date{April 24, 2025}

\begin{document}

\begin{frame}
  \titlepage
\end{frame}

\begin{frame}
  \frametitle{Main Takeaways}
  \begin{itemize}
    \setlength\itemsep{1em}
    \item Learn how to manage projects and collaborate effectively using GitHub.
    \item Explore a real-world scenario through a hands-on demo.
    \item All materials are available at: \url{https://github.com/chiatsewang/github-collaboration-tutorial}
  \end{itemize}
\end{frame}

\begin{frame}{Outline}
  \tableofcontents
\end{frame}

\section{Introduction}

\begin{frame}
  \frametitle{Introduction}
  \begin{itemize}
    \setlength\itemsep{1em}
    \item Git is a distributed version control system.
    \item Cloud-based Git services (e.g., GitHub, GitLab, Bitbucket) are used for collaboration and backup.
  \end{itemize}
\end{frame}

\begin{frame}
  \frametitle{Version Control}

  Version control helps maintain a history of changes in code and data.
  \begin{itemize}
    \setlength\itemsep{1em}
    \item No more confusing filenames like \texttt{code\_final\_v2.py} or \texttt{proj\_20250425}.
  \end{itemize}
  
  \vspace{1em}
  \textbf{Why is version control important?}
  \vspace{1em}

  \begin{enumerate}
    \setlength\itemsep{1em}
    \item Track issues and bugs; audit or revert changes when needed.
    \item Manage multiple versions of a project efficiently.
    \item Reproduce results consistently and reliably.
    \item Collaborate smoothly with others.
  \end{enumerate}
\end{frame}


\section{Git Basics}

\begin{frame}
  \frametitle{Git Overview}
  \begin{enumerate}
    \setlength\itemsep{1em}
    \item The concepts of staging and branching are fundamental in Git.
    \item A commit captures a snapshot of your project at a specific point in time.
    \item The core unit of a Git project is the repository — it stores all commits and branches.
    \item Git tracks changes in files and directories over time.
    \item Merging branches combines their changes into one unified version.
  \end{enumerate}
\end{frame}

\begin{frame}{Git Commit: Saving Your Work}
  \begin{enumerate}
    \setlength\itemsep{1em}
    \item Make changes: Edit or create files in your project.
    \item \texttt{git add}: Marks selected changes to be included in the next version.
    \item \texttt{git commit}: Saves a version of your work by recording a snapshot of the project.
    \item Each commit acts like a checkpoint — similar to saving progress in a game.
    \item You can always go back to a previous version if needed.
  \end{enumerate}
\end{frame}

\begin{frame}{Branching in Git}
  \begin{itemize}
    \item A \textbf{branch} represents an independent line of development starting from a specific commit.
    \item The default branch is usually named \texttt{main} or \texttt{master}.
    \item You can create new branches to work on features or ideas without affecting the main branch.
  \end{itemize}

  \vspace{1em}
  \textbf{Why use branches?}
  \vspace{1em}
  
  \begin{itemize}
    \item Isolate development work.
    \item Enable parallel feature development.
    \item Safely experiment with new ideas.
  \end{itemize}
\end{frame}

\begin{frame}{Git Pull: Compare and Merge Branches}
  \begin{enumerate}
    \setlength\itemsep{1em}
    \item Git allows you to combine branches that contain different work.
    \item \texttt{git pull} is a shortcut that:
    \begin{enumerate}
      \setlength\itemsep{1em}
      \item Downloads the latest changes from a target branch.
      \item Merges those changes into your current branch.
    \end{enumerate}
    \item If both branches changed the same parts of the project, Git may ask you to resolve a conflict manually.
  \end{enumerate}
\end{frame}

\begin{frame}{Forking a Git Repository}
  \begin{itemize}
    \item A \textbf{repository} stores your project’s files, commit history, and branches — like a versioned project folder.

    \item \textbf{Fork}:
    \begin{itemize}
      \setlength\itemsep{0.5em}
      \item Creates a copy of someone else’s repository under your own GitHub account.
      \item Includes the full history — all commits and branches — starting a new stream from the same point.
      \item Ideal when you don’t have direct write access — common in open source contributions.
      \item You can make changes in your fork and open a pull request to propose them to the original repository.
    \end{itemize}
  \end{itemize}
\end{frame}




\begin{comment}
\begin{frame}{Git Commit}
  \begin{itemize}
    \item \textbf{Working Directory}: Your local project folder. Files here can be edited freely.
    \item \textbf{Staging Area (Index)}: A holding area for changes you want to include in the next commit.
    \item \textbf{Repository (Local)}: A database where Git permanently stores all versions of your project.
  \end{itemize}
  \vspace{1em}
  \begin{block}{Workflow}
    \texttt{working directory} $\rightarrow$ \texttt{staging area} $\rightarrow$ \texttt{repository}
  \end{block}
  \begin{itemize}
    \item Use \texttt{git add} to move changes to staging
    \item Use \texttt{git commit} to record changes to the repository
  \end{itemize}
\end{frame}

\begin{frame}{Essential Git Commands}
  \begin{itemize}
    \item \texttt{git init}
    \item \texttt{git status}
    \item \texttt{git add}
    \item \texttt{git commit}
    \item \texttt{git log}
    \item \texttt{git diff}
  \end{itemize}
\end{frame}
\end{comment}

\section{GitHub Collaboration Best Practices}

\begin{frame}
  \frametitle{A Good Repository}
  Best practices for collaboration:
  \begin{enumerate}
    \item Document your project with a \texttt{README.md}.
    \item Define usage terms with a \texttt{LICENSE}.
    \item Use a \texttt{.gitignore} file to exclude unnecessary files.
    \item Use clear and consistent branch naming conventions.
    \item Write meaningful commit messages.
    \item Keep commits small and focused.
    \item Use pull requests for code review and discussion.
    \item Track bugs and feature requests via issues.
    \item Tag releases clearly.
  \end{enumerate}
\end{frame}

\begin{frame}{README, LICENSE, and \texttt{.gitignore}}
  \begin{itemize}
    \setlength\itemsep{1em}
    \item \texttt{README.md}: Describes the project — includes an overview, installation steps, and usage examples.
    \item \texttt{LICENSE}: Defines how the project can be used, modified, and shared (e.g., MIT, Apache 2.0, GPL).
    \item \texttt{.gitignore}: Lists files and folders Git should ignore (e.g., \texttt{*.pyc} for Python cache files, \texttt{*.log} for logs).
  \end{itemize}
\end{frame}

\begin{frame}{Branching Strategy}
  \begin{itemize}
    \setlength\itemsep{1em}
    \item \texttt{main} — the stable, production-ready branch. Avoid making direct changes here.
    \item \texttt{dev} — the primary development branch.
    \item \texttt{test} — the main testing branch used for integration and QA.
    \item Branch naming convention: \texttt{account\_name/feature/feature-name}, \texttt{account\_name/bugfix/bug-name}.
    \item Feature branches should be short-lived and focused on specific tasks.
  \end{itemize}
\end{frame}

\begin{frame}{Writing Good Commits}
  \begin{itemize}
    \setlength\itemsep{1em}
    \item Start with a clear subject line using tags like \texttt{[feat]}, \texttt{[fix]}, \texttt{[docs]}, etc.
    \item Optionally add a body for more context or explanation.
    \item Use present tense and imperative mood.
    \item Example:
    \begin{itemize}
      \setlength\itemsep{1em}
      \item \texttt{[feat] Add data loader for experiment A}
      \item \texttt{[fix] Correct typo in README}
      \item \texttt{[docs] Update installation instructions}
    \end{itemize}
  \end{itemize}
\end{frame}

\begin{frame}{GitHub Issues}
  \begin{itemize}
    \setlength\itemsep{1em}
    \item GitHub Issues are used to track bugs, feature requests, and tasks.
    \item Each issue can have a title, description, labels, and assignees.
    \item Use issues to discuss ideas and gather feedback.
    \item Link issues to commits and pull requests for better tracking.
  \end{itemize}
\end{frame}

\begin{frame}{Pull Request}
  \begin{itemize}
    \item A \textbf{Pull Request (PR)} is a request to merge one branch into another on GitHub.
    \item Typically used to propose merging a contributor’s feature branch into the main project branch.
    \item PRs enable code review, discussion, and collaboration before merging.
  \end{itemize}

  \vspace{1em}
  \textbf{Best Practices}
  \vspace{1em}
  
  \begin{itemize}
    \item Keep PRs small and focused.
    \item Provide context in the PR description.
    \item Respond to feedback promptly.
  \end{itemize}
\end{frame}



\begin{frame}{Pull Request Review Process}
  \begin{itemize}
    \setlength\itemsep{1em}
    \item PRs can be reviewed by team members or maintainers.
    \item Reviewers can comment, request changes, or approve the PR.
    \item Use comments to discuss code changes and suggest improvements.
    \item Once approved, the PR can be squashed and merged into the main branch.
    \item After merging, the feature branch can be deleted.
  \end{itemize}

\end{frame}

\section{Hands-on Demo}


\begin{frame}{Collaboration Simulation}
  % All activities will be performed using the GitHub web interface — no local Git installation required.
  
  \begin{enumerate}
    \setlength\itemsep{0.5em}
    \item Open an issue to describe the proposed change or feature.
    \item Fork the repository to your own GitHub account.
    \item Create a feature branch for your work.
    \item Make changes and commit them with clear messages.
    \item Submit a Pull Request (PR) to the original repository.
    \item The maintainer reviews the PR and may add comments or request changes.    \item Revise your code based on the feedback.
    \item The maintainer will merge the PR once it is approved.
    \item Sync your fork with the latest updates from the original repository.
  \end{enumerate}
\end{frame}

\begin{frame}{Resources}
  \begin{itemize}
    \setlength\itemsep{1em}
    \item \href{https://choosealicense.com}{choosealicense.com}
    \item \href{https://www.conventionalcommits.org/en/v1.0.0/}{conventionalcommits.org}
    \item \href{https://git-scm.com/doc}{git-scm.com/doc}
    \item \href{https://docs.github.com}{docs.github.com}
    \item \href{https://opensource.guide/}{opensource.guide}
  \end{itemize}
\end{frame}

\begin{frame}{Git \& GitHub: Recap}
  \begin{itemize}
    \setlength\itemsep{1em}
    \item Git helps you version, track, and manage changes in your projects.
    \item GitHub adds powerful collaboration features for teams and open source.
    \item Branches and pull requests support safe and structured teamwork.
    \item Clear commits, good documentation, and proper workflows lead to better collaboration.
  \end{itemize}
\end{frame}

\begin{frame}
  \vfill
  \centering
  {\LARGE \textbf{Thank you!}} \par
  \vfill

  \begin{minipage}{0.7\linewidth}
    \raggedright
    \small
    GitHub: \texttt{@chiatsewang} \\
    Email: \texttt{chiatsewang@stat.sinica.edu.tw}
  \end{minipage}
\end{frame}
\end{document}